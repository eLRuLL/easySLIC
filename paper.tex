\documentclass[11pt, oneside]{article}   	% use "amsart" instead of "article" for AMSLaTeX format
\usepackage{geometry}                		% See geometry.pdf to learn the layout options. There are lots.
\geometry{letterpaper}                   		% ... or a4paper or a5paper or ... 
%\geometry{landscape}                		% Activate for for rotated page geometry
%\usepackage[parfill]{parskip}    		% Activate to begin paragraphs with an empty line rather than an indent
\usepackage{graphicx}				% Use pdf, png, jpg, or eps§ with pdflatex; use eps in DVI mode
								% TeX will automatically convert eps --> pdf in pdflatex		
\usepackage{caption}
\usepackage{subcaption}
\usepackage{amssymb}
\usepackage{amsmath}
\usepackage{algorithm}
%\usepackage{algorithmic}
\usepackage{algpseudocode}


\title{Implementaci\'on del algoritm gSLIC}
\author{Raul Gallegos, Juan Carlos Due\~nas}
\date{\today}							% Activate to display a given date or no date
\begin{document}
\maketitle
\begin{abstract}
En este trabajo, se presenta la implementaci\'on del algoritmo ``gSLIC: a real-time implementation of SLIC superpixel segmentation'' de Carl Yuheng Ren and Ian Reid\cite{YHRen_gSLIC}.
El algoritmo gSLIC introduce una implementaci\'on paralela de la segmentaci\'on SLIC. Esta implementaci\'on usa GPU con el framework de NVIDIA CUDA. Junto con la implementaci\'on en este trabajo tambi\'en presentaremos un video que demuestra como se aplica a un ambiente en tiempo real. 
\end{abstract}

\section{Introduction}
El algoritmo Simple Linear Iterative Clustering (SLIC) para segmentaci\'on de superpixeles est\'a basado en el algoritmo de k-means para clustering de pixeles en un espacio en 5-d (labxy) definido por los valores L, a, b del espacio de color CIELAB y las coordenadas del pixel $x,y$. El espacio CIELAB fue escogido ya que se percibe mejor para distancias pequeñas de colores. En vez de usar la distancia euclidiana directamente en el espacio 5-d, se introduce la siguiente función de distancia entre pixeles.
$$
d_{lab}=\sqrt{(l_k-l_i)^2+(a_k-a_i)^2+(n_k-b_i)^2}
$$
$$
d_{xy}=\sqrt{(x_k-x_i)^2+(y_k-y_i)^2}
$$
$$
D_s=d_{lab}+\frac{m}{S}d_{xy}
$$

Donde $S = \sqrt{\frac{N}{K}}$, $N=$ n\'umero de pixeles en la imagen y $K=$ n\'umero de superpixeles deseados (par\'ametro). El par\'ametro $m$ es introducido para controlar cuan compactos son los superpixeles. Mientras m\'as grande sea el $m$, m\'as compacto el cluster. A continuaci\'on se presenta el algoritmo modificado SLIC para que funcione cada thread con un pixel (Algoritm \ref{algo}).
\begin{algorithm}
\caption{Algoritmo SLIC modificado}
\label{algo}
\begin{algorithmic}[1]
\State{Inicializar centros de los clusters $[l_k,a_k,b_k,x_k,y_k]^T$ con pixeles ejemplo cada $S$ pixeles.}
\State{Perturbar los centros de los clusters hacia la posición de menor gradiente}
\For{cada pixel}
    \State{Asigna el pixel al cluster m\'as cercano}
\EndFor
\While{N\'umero de iteraciones}
\For{cada pixel}
\State{Localmente busca la m\'as cercana de 9 cl\'usteres vecinos}
\State{etiqueta este pixel con el cluster m\'as cercano}
\EndFor
\State{Actualiza cada centro de cluster basado en asignamiento de pixeles.}
\EndWhile
\State{Forzar conectividad}
\end{algorithmic}
\end{algorithm}

\section{Resultados}

Primero para probar la t\'ecnica y asegurarnos que el algoritmo funcionaba antes de hacerlo en GPU, probamos el algoritmo SLIC en CPU con diferentes im\'agenes. Mostramos resultados en las Figuras 1,2 y 3.


\begin{figure}
\label{fig1}
\centering
\begin{subfigure}[b]{0.49\textwidth}
\label{dog1sub1}
\caption{Dog Image: 1280 x 854 pixels}
\includegraphics[width=\textwidth]{dog_high}
\end{subfigure}
\begin{subfigure}[b]{0.49\textwidth}
\label{dog1sub2}
\caption{Imagen con superpixeles: 100 superpixeles y m=40}
\includegraphics[width=\textwidth]{super_dog_high}
\end{subfigure}
\caption{[SLIC] La generaci\'on de superpixeles demor\'o 13.2884 seg.}
\end{figure}

\begin{figure}
\label{fig2}
\centering
\begin{subfigure}[b]{0.49\textwidth}
\label{dog2sub1}
\caption{Dog Image: 711 x 474 pixels}
\includegraphics[width=\textwidth]{dog_high2}
\end{subfigure}
\begin{subfigure}[b]{0.49\textwidth}
\label{dog2sub2}
\caption{Imagen con superpixeles: 100 superpixeles y m=40}
\includegraphics[width=\textwidth]{super_dog_high2}
\end{subfigure}
\caption{[SLIC] La generaci\'on de superpixeles demor\'o 4.21426 seg.}
\end{figure}


\begin{figure}
\label{fig3}
\centering
\begin{subfigure}[b]{0.49\textwidth}
\label{dog3sub1}
\caption{Dog Image: 178 x 119 pixels}
\includegraphics[width=\textwidth]{dog_high3}
\end{subfigure}
\begin{subfigure}[b]{0.49\textwidth}
\label{dog3sub2}
\caption{Imagen con superpixeles: 100 superpixeles y m=40}
\includegraphics[width=\textwidth]{super_dog_high3}
\end{subfigure}
\caption{[SLIC] La generaci\'on de superpixeles demor\'o 0.273571 seg.}
\end{figure}

%\subsection{}


\bibliography{mybib}{}
\bibliographystyle{plain}
\end{document}  